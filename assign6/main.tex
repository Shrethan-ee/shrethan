\let\negmedspace\undefined
\let\negthickspace\undefined
\documentclass[journal,12pt,twocolumn]{IEEEtran}
\usepackage{cite}
\usepackage{amsmath,amssymb,amsfonts,amsthm}
\usepackage{algorithmic}
\usepackage{graphicx}
\usepackage{textcomp}
\usepackage{xcolor}
\usepackage{txfonts}
\usepackage{listings}
\usepackage{enumitem}
\usepackage{mathtools}
\usepackage{gensymb}
\usepackage{comment}
\usepackage[breaklinks=true]{hyperref}
\usepackage{tkz-euclide} 
\usepackage{listings}
\usepackage{gvv}                                        
%\def\inputGnumericTable{}                                 
\usepackage[latin1]{inputenc}                                
\usepackage{color} 
\usepackage{array}
\usepackage{longtable}
\usepackage{calc}   
\usepackage{multirow}
\usepackage{hhline}
\usepackage{ifthen}
\usepackage{lscape}
\usepackage{tabularx}
\usepackage{array}
\usepackage{float}

\newtheorem{theorem}{Theorem}[section]
\newtheorem{problem}{Problem}
\newtheorem{proposition}{Proposition}[section]
\newtheorem{lemma}{Lemma}[section]
\newtheorem{corollary}[theorem]{Corollary}
\newtheorem{example}{Example}[section]
\newtheorem{definition}[problem]{Definition}
\newcommand{\BEQA}{\begin{eqnarray}}
\newcommand{\EEQA}{\end{eqnarray}}
\newcommand{\define}{\stackrel{\triangle}{=}}
\theoremstyle{remark}
\newtheorem{rem}{Remark}

% Marks the beginning of the document
\begin{document} 

\bibliographystyle{IEEEtran}
\vspace{3cm}

\title{07-26-2022 SHIFT-1-16-30}
%{$2022-$July SESSION $-07-26-2022$ SHIFT$-1$}
\author{EE24BTECH11029- JANAGANI SHRETHAN REDDY}
\maketitle{}
\newpage
\bigskip
\renewcommand{\thefigure}{\theenumi}
\renewcommand{\thetable}{\theenum}
\begin{enumerate}
    \item The mean and variance of a binomial distribution are $\alpha$ and $\frac{\alpha}{3}$ respectively. If $P\brak{X=1}=\frac{4}{243},$ then $P\brak{X=4 or 5}$ is equal to:
    \begin{enumerate}
        \item $\frac{5}{9}$
        \item $\frac{64}{81}$
        \item $\frac{16}{27}$
        \item $\frac{145}{243}$\\
    \end{enumerate}
    \item Let $E_1,E_2,E_3$ be three mutually exclusive events such that $P\brak{E_1}=\frac{2+3p}{6},P\brak{E_2}=\frac{2-p}{8}$ and $P\brak{E_3}=\frac{1-p}{2}.$ If the maximum and minimum values of $p$ are $p_1$ and $p_2,$ then $\brak{p_1+p_2}$ is equal to :
    \begin{enumerate}
        \item $\frac{2}{3}$
        \item $\frac{5}{3}$
        \item $\frac{5}{4}$
        \item $1$\\
    \end{enumerate}
    \item Let $S\{\theta\in\sbrak{0,2\pi};8^{2\sin^2\theta}+8^{2\cos^2\theta}=16\}.$ Then $n\brak{S}+\sum_{\theta\in S}\brak{\sec\brak{\frac{\pi}{4}+2\theta}\cosec\brak{\frac{\pi}{4}+2\theta}}$ is equal to :
    \begin{enumerate}
        \item $0$
        \item $-2$
        \item $-4$
        \item $12$\\
    \end{enumerate}
    \item $\tan\brak{2\tan^{-1}\frac{1}{5}+\sec^{-1}\frac{\sqrt{5}}{2}+2\tan^{-1}\frac{1}{8}}$ is equal to:
    \begin{enumerate}
        \item $1$
        \item $2$
        \item $\frac{1}{4}$
        \item $\frac{5}{4}$\\
    \end{enumerate}
    \item The statement $\brak{\sim\brak{p\Leftrightarrow\sim q}}\land q$ is:
    \begin{enumerate}
        \item a tautology
        \item a contradiction
        \item equivalent to $\brak{p\implies q}\land q$
        \item equivalent to $\brak{p\implies q}\land p$\\
    \end{enumerate}
%\maketitle{SECTION-B}
    \item If for some $p,q,r\in R$,not all have same  sign, one of the roots of the equation $\brak{p^2+q^2}x^2-2q\brak{p+r}x+q^2+r^2=0$ is also a root of the equation $x^2+2x-8=0,$ then $\frac{q^2+r^2}{p^2}$ is equal to\\
    \item The number of $5-$digit natural numbers, such thet the product of their digits is $36,$ is\\
    \item The series of positive multiple of $3$ is divided into sets: $\{3\},\{6,9,12\},\{15,18,21,24,27\},\dots$ Then the sum of the elments in the $11^{th}$ set is equal to\\
    \item The number of distinct real of the equation $x^5\brak{x^3-x^2-x+1} +x\brak{3x^3-4x^2-2x+4}-1=0$ is\\
    \item If the coefficients of $x$ and $x^2$ in the expansion of $\brak{1+x}^p\brak{1-x}^q,p,q\le 15,$ are $-3$ and $-5$ respectively, then the coefficient of $x^3$ is equal to.\\
    \item If $n\brak{2n+1}\int_{0}^{1}\brak{1-x^n}^{2n}\,dx=1177\int_{0}^{1}\brak{1-x^n}^{2n+1}\,dx,$ then $n\in N$ is equal to\\
    \item Let a curve $y=y\brak{x}$ pass through the point $\brak{3,3}$ and the area of the region under this curve, above the $x-$axis and between the abscissae $3$ and $x\brak{\textgreater 3}$ be $\brak{\frac{y}{x}}^3.$ If this curve also passes through the point\\ $\brak{\alpha,6\sqrt{10}}$ in the first quadrant, then $\alpha$ is equal to\\
    \item The equations of the sides $AB,BC$ and $CA$ of a triangle $ABC$ are $2x+y=0,x+py=15a$ and $x-y=3$ respectively. If its orthocentre is $\brak{2,a},-\frac{1}{2} \textless a \textless 2,$ then $p$ is equal to\\
    \item Let the function $f\brak{x}=2x^2-\log_e{x},x\textgreater0,$ be decreasing in $\brak{0,a}$ and increasing in $\brak{a,4}$.A tangent to the parabola $y^2=4ax$ at a point $P$ on it passes through the point $\brak{8a,8a-1}$ but does not pass through the point $\brak{-\frac{1}{a},0}.$ If the equation of the normal at $P$ is $\frac{x}{\alpha}+\frac{y}{\beta}=1,$ then $\alpha+\beta$ is equal to\\
    \item Let $Q$ and $R$ be two points on the line $\frac{x+1}{2}=\frac{y+2}{3}=\frac{z-1}{2}$ at a distance $\sqrt{26}$ from the point $P\brak{4,2,7}.$Then the square of the area of the $PQR$ is
\end{enumerate}
\end{document}
